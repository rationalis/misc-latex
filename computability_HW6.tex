\documentclass[letterpaper,12pt]{article}
\usepackage{fullpage}
\usepackage{amssymb}
\usepackage{hyperref}

\def\name{Jimmy Ye}
\def\id{A12405090}

\newtheorem{theorem}{Theorem}
\newenvironment{proof}{\emph{Proof:}}{\hfill $\Box$}

\begin{document}

\newlength{\boxwidth}
\setlength{\boxwidth}{\textwidth}
\addtolength{\boxwidth}{-2cm}
\noindent\framebox[\textwidth]{\hfil
\parbox[t]{\boxwidth}{%
{\bfseries CSE\,105: Automata and Computability Theory \hfill Fall 2015}
\begin{center}\huge Homework \#6\end{center}
\name\hfill\id%
}\hfil}
\vspace{0.7cm}

\noindent\textbf{Solution to Problem 1}

1. $L_{NFA} \leq_m L_{NFA1}$

Yes, $f$ is a valid mapping reduction.

\vspace{2mm}

2. $L_{NFA} \leq_m ONE_{NFA}$

No, $f$ is not a valid mapping reduction. Consider any NFA accepting more than one string (for example, the trivial NFA accepting every string formed from the alphabet $\{a\}$). Then $f$ maps this NFA, $M$, to an equivalent NFA, $M'$. $\langle M \rangle$ is in $L_{NFA}$ by definition, but $M'$ accepts more than one string, so $\langle M' \rangle$ is not in $ONE_{NFA}$.

\vspace{2mm}

3. $EPS_{NFA} \leq_m EPS_{NFA}$

Yes, $f$ is a valid mapping reduction.

\vspace{2mm}

4. $ONE_{NFA} \leq_m ONE_{NFA1}$

Yes, $f$ is a valid mapping reduction.

\vspace{2mm}

5. $EPS_{NFA} \leq_m ONE_{NFA1}$

No, $f$ is not a valid mapping reduction. Consider any NFA which accepts only one non-empty string (for example, the trivial NFA over the alphabet $\{a\}$ accepting only $a$). Then $f$ maps this NFA, $M$, to an equivalent NFA, $M'$. However, $\langle M \rangle$ is not in $EPS_{NFA}$ since $M$ accepts a non-empty string, but $\langle M' \rangle$ is in $ONE_{NFA1}$ since $M'$ accepts only one string and has only one final state.

\vspace{2mm}

6. $ONE_{NFA} \leq_m L_{NFA}$

No, $f$ is not a valid mapping reduction. Consider any NFA $M$ accepting more than one string. $M$ is not in $ONE_{NFA}$, but $f$ maps $M$ to an equivalent NFA $M'$ and $M'$ is in $L_{NFA}$.

\pagebreak

\noindent\textbf{Jimmy Ye | A12405090}

\noindent\textbf{Solution to Problem 2}

a. Prove that $ONE_{TM}$ is not recognizable by mapping-reduction from the diagonal language $D = \{\langle M \rangle | M$ is a Turing machine such that $\langle M \rangle \notin L(M)\}$.

\vspace{3mm}

We construct such a mapping-reduction, $f$, which outputs $\langle M' \rangle$:

Given any word $w$, if it is not a valid encoding of a Turing machine, $M'$ always rejects.

Otherwise, $w = \langle M \rangle$ for some Turing machine $M$. We treat Turing machines as the languages which they accept, taking union and intersection as operations which construct Turing machines, which simulate both operands and accept the union or intersection of their languages, respectively.

Construct $M_1$ as a Turing machine accepting only $\langle M \rangle$, and $M_2$ as a Turing machine accepting only $\epsilon$, the empty string. Then we construct $M'$ as $M_2 \cup (M \cap M_1)$, which is possible in finite time using only the encoding of $M$.

If $\langle M \rangle \in D$: $M_2$ accepts $\epsilon$. $M$ does not accept $\langle M \rangle$ and by construction $M_1$ accepts no other strings, so $M \cap M_1$ accepts no strings. So $M'$ accepts only one string, $\epsilon$, so $\langle M' \rangle \in ONE_{TM}$.

If $\langle M \rangle \notin D$: $M_2$ accepts $\epsilon$. $M$ and $M_1$ accept $\langle M \rangle$, so $M \cap M_1$ accepts $\langle M \rangle$. So $M'$ accepts $\epsilon$ and $\langle M \rangle$, so $\langle M' \rangle \notin ONE_{TM}$.

\vspace{3mm}

b. Prove that $ONE_{TM}$ is not co-recognizable by mapping-reduction from the complement of the diagonal language $\overline{D}$.

\vspace{3mm}

Note that $\overline{D} = \{\langle M \rangle | \langle M \rangle \in L(M)\}$

We construct such a mapping-reduction, $f$, which outputs $\langle M' \rangle$:

Given any word $w$, if it is not a valid encoding of a Turing machine, $M'$ always rejects. Otherwise, $w = \langle M \rangle$ for some Turing machine $M$. Construct $M_1$ as a Turing machine accepting only $\langle M \rangle$. Then we construct $M'$ as $M \cap M_1$, which is possible in finite time using only the encoding of $M$.

If $\langle M \rangle \in \overline{D}$, $M$ and $M_1$ both accept $\langle M \rangle$, and by construction $M_1$ accepts no other strings. So $M'$ accepts only one string, $\langle M \rangle$, so $\langle M' \rangle \in ONE_{TM}$.

If $\langle M \rangle \notin \overline{D}$, $M$ does not accept $\langle M \rangle$, and by construction $M_1$ accepts no other strings. So $M'$ accepts no strings, so $\langle M' \rangle \notin ONE_{TM}$.

\pagebreak

\noindent\textbf{Jimmy Ye | A12405090}

\noindent\textbf{Solution to Problem 3}

a. Give a mapping reduction from $ONE_{TM}$ to $TWO_{TM}$

\vspace{3mm}

We construct such a mapping-reduction, $f$, which outputs $\langle M' \rangle$:

Again, given $w$, if $w$ is not a valid Turing machine encoding, $M'$ always rejects.

Otherwise, $w = \langle M \rangle$ for some Turing machine M. We construct $M'$ from $M$, adding an additional symbol $x$ to the alphabet of $M'$ which is not in the alphabet of $M$. We construct $M'$ to accept the string $x$, but otherwise reject strings containing an $x$ symbol. For strings that do not contain an $x$, $M'$ behaves identically to $M$.

Then $M'$ accepts exactly one more string than $M$ does. So if $\langle M \rangle \in ONE_{TM}$, then $\langle M' \rangle \in TWO_{TM}$, and likewise if $\langle M \rangle \notin ONE_{TM}$, then $\langle M' \rangle \notin TWO_{TM}$.

\vspace{3mm}

b. Give a mapping reduction from $TWO_{TM}$ to $ONE_{TM}$

\vspace{3mm}

We construct such a mapping-reduction, $f$, which outputs $\langle M' \rangle$:

Again, given $w$, if $w$ is not a valid Turing machine encoding, $M'$ always rejects.

Otherwise, $w = \langle M \rangle$ for some Turing machine $M$. Since $M$ has a finite alphabet, we assign some arbitrary ordering of this alphabet (and subsequently this generates a lexicographical ordering for every string we can generate from this alphabet).

Then on any input $w$, we construct $M'$ to simultaneously run $M$ on every possible input $w'$ which comes before $w$ in the lexicographical ordering, as well as $w$ itself.

If $M$ accepts $w$ before any other $w'$ (i.e. at the step at which $M(w)$ reaches $q_{accept}$, no other simulated instance of $M$ has reached $q_{accept}$), then $M'$ rejects. If $M$ accepts $w$ at the same step as a nonempty set of other $w'$, then $M'$ rejects $w$ only if $w$ is the lexicographically smallest word, and accepts otherwise. If $M$ accepts $w$ after any other $w'$, then $M'$ accepts $w$.

Since, for any given $w$, there are finitely many $w'$ which are lexicographically lesser, $M'$ simulates finitely many instances of $M$, so $M'$ is well-defined. Lexicographical comparison can be computed by a Turing machine in a finite number of steps, and there are finitely many $w'$, so checking if $w$ is the smallest can be done in a finite number of steps.

So by construction, $M'$ differs from $M$ only when the input $w$ is the unique word (1) accepted in the least number of steps by $M$ and (2) the lexicographically smallest such word if multiple words are accepted in the same number of steps. In this case, $M'$ rejects $w$ whereas $M$ accepts $w$.

So $M'$ always accepts one less string than $M$ (unless $M$ accepts no strings, in which case $M'$ also accepts no strings), so if $\langle M \rangle \in TWO_{TM}$, then $\langle M' \rangle \in ONE_{TM}$, and likewise if $\langle M \rangle \notin TWO_{TM}$, then $\langle M' \rangle \notin ONE_{TM}$

\pagebreak

\noindent\textbf{Jimmy Ye | A12405090}

\noindent\textbf{Solution to Problem 4}

Use the results from problem 2 and problem 3 to prove that the language $TWO_{TM}$ is \textit{not} Turing equivalent to the diagonal language D.

\vspace{3mm}

From (3), we know that $ONE_{TM}$ and $TWO_{TM}$ are Turing equivalent. From (2), we know that $ONE_{TM}$ is neither recognizable nor co-recognizable. This implies that $TWO_{TM}$ is neither recognizable nor co-recognizable.

Consider $\overline{D}$. It is recognizable, since given $w = \langle M \rangle$, we construct $M'$ to compute $M$ with input $\langle M \rangle$, and if it is in $\overline{D}$, it will accept, and if it is not in $\overline{D}$, it will not accept. (On input $w$ not representing a Turing machine, $M'$ rejects)

Assume that $TWO_{TM}$ is Turing equivalent to $D$. Then since $TWO_{TM}$ is neither recognizable nor co-recognizable, $D$ is also neither recognizable nor co-recognizable. But if $D$ is not co-recognizable, then $\overline{D}$ is not recognizable, which contradicts what we have shown above.

So $TWO_{TM}$ is not Turing equivalent to $D$.

\end{document}
